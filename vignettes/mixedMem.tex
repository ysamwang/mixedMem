%\VignetteEngine{knitr::knitr}
%\VignetteIndexEntry{mixedMem}
\documentclass{article}

\linespread{1.6}
\usepackage{amsmath,amsthm,amssymb}
\usepackage[margin = 1in]{geometry} 
\usepackage{natbib, underscore}
\newcommand{\E}{\mathbb{E}}
\usepackage{setspace} 
\renewenvironment{knitrout}{\begin{singlespace}}{\end{singlespace}}


\usepackage{Sweave}
\begin{document}
\Sconcordance{concordance:mixedMem.tex:mixedMem.Rnw:%
1 84 1 48 0 1 3 15 1 24 0 44 1 4 0 12 1 21 0 7 1 29 0 20 1 13 0 10 %
1 22 0 12 1 13 0 14 1 13 0 12 1 12 0 7 1 4 0 7 1 9 0 20 1 18 0 3 1 %
1 3 9 1 7 0 26 1 13 0 20 1}

\title{Fitting Mixed Membership Models using \texttt{mixedMem}\footnotetext{Special thanks to McKay Curtis, Maryclare Griffin and Kevin Lin for providing feedback on package features and usability}}
\author{Y. Samuel Wang and Elena A. Erosheva}
\maketitle

\small
\begin{abstract}
This vignette is a tutorial for the \texttt{mixedMem} R package that allows users to fit and visualize results for a broad class of mixed membership models as specified by \cite{erosheva2004mixed}. Mixed distribution types and replicate measurements are supported. Version 1.1.0 allows for binary, multinomial or Plackett-Luce rank distribution types and the extended GoM model from \cite{erosheva2007describing} is also supported. This tutorial provides an outline of functions available in the package and a step-by-step guide for fitting mixed membership models. As examples, we carry out a mixed membership analysis to identify latent ideology blocs using political survey data from the 1983 American National Election Survey Pilot as well as UPDATE  
\end{abstract}

\normalsize
\section{Mixed Membership Modeling} \label{MMM}
\subsection{Mixed Membership Background}
Mixed membership models provide a useful framework for analyzing multivariate data from heterogeneous populations \citep{Airoldi2014Handbook}. Similar to mixture models, mixed membership models assume that the overall population is comprised of several latent sub-populations or groups. Mixture models, however, assume that each individual within the population belongs to a single group, whereas mixed membership models allow for an individual to belong to multiple groups simultaneously and explicitly specify an individual's degree of membership in each group.

Mixed membership models have been used to analyze to a wide variety of data types. For example, mixed membership analysis of text data \citep{LDA, erosheva2004mixed} can illustrate the latent structure of topics/subjects within a body of documents; analysis of genotype sequences \citep{pritchard2000inference} can describe individuals sharing ancestry across various origin populations; analysis of survey data \citep{erosheva2007describing, grossManriqueVallier} can show how survey respondents belong to various sub-populations; and analysis of ranked votes \citep{gormley2009grade} can characterize latent voting blocs and voting tendencies. The \texttt{mixedMem} package allows for fitting these specific cases of mixed membership models as well as their extensions, using a general framework described by \cite{erosheva2004mixed} with the estimation carried out via a variational EM algorithm.

The remainder of section \ref{MMM} introduces notation and formally defines the generative hierarchy assumed for discrete mixed membership models. Section \ref{variational} outlines the variational EM  method used by the \texttt{mixedMem} package and discusses some of the benefits and drawbacks of the variational approach. Section \ref{politicalSurvey} provides a step-by-step guide for fitting mixed membership models using \texttt{mixedMem}. For illustration, we analyze the results of the 1983 American National Election Studies Pilot Study. Finally, section \ref{conclusion} provides a brief conclusion.

\subsection{Illustration and Notation}
For a hypothetical example, consider a survey of Total =  100 (the number of individuals) high school students with J = 3 (the number of variables) questions. Each individual is asked to (1) rank their favorite classes (rank data), (2) select their favorite TV channel (multinomial data) and (3) indicate whether or not they are on the honor roll (Bernoulli data). Students within the same extracurricular activities/clubs might give similar responses, so in a mixed membership analysis, the latent sub-populations might map to extracurricular groupings. For concreteness, assume the following K = 4 sub-populations fully describe the students' extracurricular activities: athletics, math/science, fine arts and student government. Athletes might be more likely to rank ``Anatomy" and ``Health" among their favorite classes, while students involved in student government might be more likely to rank ``Civics" or ``History" highly.  As is often the case, students may be involved with multiple clubs/activities to varying degrees. If a student is involved in student government and athletics, they might rank their favorite classes as (1. ``Anatomy", 2. ``History" and 3. ``Health"). Thus, using a mixture model framework to classify the student solely as an athlete would fail to represent her full profile. We use $\lambda_i$ to denote the distribution of membership for individual i. $\lambda_i$ is a vector of K non-negative elements which sum to 1. The membership vector $\lambda$ for the student above might be $\left(\text{athletics} = 0.7, \text{math/science} = 0.0, \text{fine arts} = 0.0, \text{student government} = 0.3\right)$. In this package, we assume that memberships are drawn from a Dirichlet distribution; however, some other mixed membership models assume more general distributions for the membership vectors \citep{blei2006correlated}.

We index each of the Total students by $i = 1,2, \ldots Total$; in this case Total = 100. Each of the J variables are indexed by $j = 1,2,\ldots J$ and for each variable we index the $R_j$ number of replicates by $r = 1,2,\ldots R_j$. Note that each of the variables can have a different number of replications. For rank data, we index the $N_{i,j,r}$ ranking levels (ie the first preference, the second preference, ...) by $n = 1,2\ldots N_{i,j,r}$; for multinomial and binary data $N_{i,j,r} = 1$. For each individual i, variable j, replicate r, and ranking level n we denote the observed response by $X_{i,j,r,n}$. For example, if student 10's favorite classes (variable 1)  are $ \left(\text{``Anatomy", ``History", ``Health"}\right)$ and she is on the honor roll (variable 3), then $X_{i = 10,j = 1,r = 1,n = 2} =$ ``History" and  $X_{i = 10,j = 3,r = 1,n = 1} = 1$.    

\subsection{Generative Process}\label{generative}
More formally, for K sub-populations, we assume a mixed membership generative process as follows:

\noindent
For each individual $i = 1,\ldots Total$: Draw $\lambda_i$ from a Dirichlet($\alpha$). $\lambda_i$ is a vector of length K which indicates the degree of membership for individual i.
  \begin{itemize}
  \item For each variable $j = 1, \ldots, J$, each replicate $r = 1, \dots, R_j$ and each ranking level $n = 1\ldots, N_{i,j,r}$: Draw $Z_{i,j,r,n}$ from a multinomial(1, $\lambda_i$). $Z_{i,j,r,n}$ determines the sub-population which governs the response for observation $X_{i,j,r,n}$. This is sometimes referred to as a context vector because it determines the context from which the individual responds.
  \item For each variable $j = 1, \ldots, J$, each replicate $r = 1, \dots, R_j$ and each ranking level $n = 1\ldots, N_{i,j,r}$: Draw $X_{i,j,r,n}$ from the distribution parameterized by $\theta_{j,Z_{i,j,r,n}}$. Here, $\theta$ is the set of parameters which govern the observations for each sub-population. If variable j is a multinomial or rank distribution with $V_j$ alternative, $\theta_{j,k}$ is a vector of length $V_j$ which parameterizes the responses to variable j for sub-population k. If variable j is a Bernoulli random variable, then $\theta_{j,k}$ is a value which determines the probability of success. 
  \end{itemize}
  
In topic models, \cite{LDA, erosheva2004mixed} analyze individual articles; the latent sub-populations are topics for each of the articles; the the observed words are draws from a multinomial distribution determined by the topic; and we observe many replicates (words) for each article . In the analysis of genotype sequences, \cite{pritchard2000inference} analyze individuals; the latent sub-populations map to genetically similar groups; each observed genetic loci is a different variable; and there are two replicates (assuming diploid individuals) for each variable. In the analysis of ranked votes, \cite{gormley2009grade} analyze individual voters; the latent sub-populations map to voting blocs; the observed data the a list of candidates by preference for each individual; and the ranking levels ($N_{ijr}$) for each individual is the number of candidates they list.     

\section{Variational EM and the \texttt{mixedMem} Package}\label{variational}

\subsection{Fitting Mixed Membership Models}
When using a mixed membership model, the interest is typically in estimating the sub-population parameters $\theta$, the Dirichlet parameter $\alpha$ and the latent memberships of individuals $\lambda$. Estimation of these quantities through maximum likelihood or direct posterior inference is computationally intractable in a mixed membership model, so Markov Chain Monte Carlo or variational inference techniques are used instead \citep{airoldi2009mixed}. Most MCMC analyses typically require large amounts of human effort to tune and check the samplers for convergence. Furthermore, in mixed membership models, we must sample a latent membership for each individual and a context vector for each observed response; thus, a mixed membership MCMC analysis becomes very computationally expensive as the number of individuals grows. \texttt{mixedMem} circumvents these difficulties by employing a variational EM algorithm which is a deterministic and computationally attractive alternative for fitting mixed membership models. Using a variational approach allows us to fit larger models and avoids tedious human effort by approximating the true posterior and replacing the MCMC sampling procedure with an optimization problem \citep{beal2003variational}.  

\subsection{Variational EM}\label{VI}
To fit a mixed membership model, the variational EM algorithm combines variational inference (inference on an approximate posterior) to estimate the group memberships $\lambda$ and a pseudo maximum likelihood procedure to estimate the group parameters $\theta$ and Dirichlet parameter $\alpha$.

Instead of working with the intractable true posterior, variational inference employs a more computationally tractable approximate variational distribution. This variational distribution, denoted by Q, is parameterized by $\phi$ and $\delta$ as follows: 

\begin{equation} \label{eq:varDist}
p(\lambda, Z|X) \approx Q(\lambda,Z|\phi, \delta) = \prod_i^T \text{Dirichlet}(\lambda_i|\phi_i)\prod_j^J \prod_r^{R_j} \prod_n^{N_{i,j,r}}\text{multinomial}(Z_{i,j,r,n}|\delta_{i,j,r,n}).
\end{equation}

The parameters $\phi$ and $\delta$ can be selected to minimize the Kullback-Leibler divergence between the true posterior distribution and the variational distribution Q \citep{beal2003variational}. This provides an approximate distribution which can be used to carry out posterior inference on the latent variables and posterior means which can be used as point estimates for $\lambda$.

This variational distribution can also be used in an alternative pseudo-likelihood framework, to select the global parameters $\theta$ and $\alpha$. Using Jensen's inequality, the variational distribution can be used to derive a function of $\phi$, $\delta$, $\alpha$ and $\theta$ which is a lower bound on the log-likelihood of our observations:

\begin{equation}\label{eq:Jensen}
\log\left[p(X|\alpha, \theta)\right] \geq \E_Q\left\{\log\left[p(X,Z, \Lambda)\right]\right\} - \E_Q\left[\log\left[Q(Z, \Lambda|\phi, \delta)\right]\right\}.
\end{equation}

The lower bound on the RHS of equation (\ref{eq:Jensen}) is often called the ELBO for \textbf{E}vidence \textbf{L}ower \textbf{Bo}und. Calculating the LHS of equation (\ref{eq:Jensen}) is intractable, but for a fixed $\phi$ and $\delta$, selecting $\alpha$ and $\theta$ to maximize the lower bound on the RHS is a tractable alternative to maximum likelihood estimation. 

It can be shown that minimizing the KL Divergence between Q and the true posterior is actually equivalent to maximizing the lower bound in equation (\ref{eq:Jensen}) with respect to $\phi$ and $\delta$ \citep{beal2003variational}. Thus, the tasks of finding an approximate posterior distribution with respect to $\phi$ and $\delta$ and picking pseudo-MLE's $\theta$ and $\alpha$ are both achieved by maximizing the lower bound in equation (\ref{eq:Jensen}). In practice, we maximize this lower bound through a variational EM procedure. In the E-step (variational inference), we fix $\theta$ and $\alpha$ and minimize the KL divergence between Q and the true posterior   (this is also equivalent to maximizing the lower bound in equation (\ref{eq:Jensen}) through iterative closed form updates to $\phi$ and $\delta$. In the M-step (pseudo-MLE), we fix $\phi$ and $\delta$ and select the $\theta$ and $\alpha$ which maximize the lower bound through gradient based methods. The entire procedure iterates between the E-step and the M-step until reaching a local mode. A detailed exposition of variational inference is provided by \cite{jaakkola200110}.

\subsubsection{Label Switching in Mixed Membership Models}
Mixed membership models are only identifiable up to permutations of the sub-population labels (ie simultaneously permuting the labels for all group memberships and distribution parameters). In an MCMC analysis of mixed membership models, this can require special attention if label switching is present within a sampler \citep{stephens2000dealing}. Because variational EM is a deterministic approach, the final permutation of the labels is only dependent on the initialization points and does not require special attention during the fitting procedure. However matching group labels can still a concern for assessing model fit when comparing to some other fitted model. \texttt{mixedMem} provides functions discussed in subsection \ref{postProcess}  for dealing with the label permutations to facilitate post-processing.   

\subsubsection{Variational EM Caveats}
The computational benefits of variational EM, however, do not come for free. The lower bound in equation (\ref{eq:Jensen}), is generally not a strictly convex function, so only convergence to a local mode, not the global mode, is guaranteed \citep{wainwright2008graphical}. If prior knowledge exists about a specific problem, initializing $\theta$ and $\alpha$ near plausible values is helpful in ensuring that the EM algorithm reaches a reasonable mode. In general though, starting from multiple initialization points and selecting the mode with the largest ELBO is highly recommended. This is also an area where the post-processing tools discussed in subsection \ref{postProcess} can be useful for determining if candidate modes found from different initialization points are conceptually different.

Variational inference also lacks any guarantees on the quality of our approximation. In some cases, we anecdotally find that the estimates of $\alpha$ provide good estimates of relative frequency, but can inaccurately estimate the rate of intra-individual dispersion (which is regulated by $\sum \alpha_i$). However, \cite{erosheva2007describing} show that a mixed membership analysis of survey data using a MCMC and variational approach agree well, and, in practice, we see that variational inference provides reasonable and interpretable results in mixed membership models \citep{LDA, erosheva2004mixed, airoldi2009mixed}. However, there are no theoretical guarantees on how good or bad our approximation ultimately is.

\section{Example: Fitting Political Survey Data with \texttt{mixedMem}} \label{politicalSurvey}
For demonstration, we present a \texttt{mixedMem} analysis of political opinion survey data previously analyzed by  \cite{grossManriqueVallier} as well as \cite{feldman1988structure}. Within this context, identified latent sub-populations might map to ideological blocs. Since individuals often hold to political ideologies to varying degrees, a mixed membership model is particularly appropriate. We utilize the mixed membership model as specified by Gross and Manrique-Vallier and discussed more generally in section \ref{generative}. The model assumes 3 latent sub-populations (K = 3) with 19 observed multinomial variables with 1 replicate each. Gross and Manrique-Vallier specify a fully Bayesian approach, placing prior distributions on both $\theta$ and $\alpha$ and utilize MCMC to estimate the model. This allows for posterior inference on both the latent memberships as well as on $\theta$ and $\alpha$. Our analysis using \texttt{mixedMem} will fit the model using the variational EM algorithm  which allows for posterior inference on $\lambda$ and Z, but only yields point estimates for $\alpha$ and $\theta$. Nonetheless, we will show that the two methods yield comparable results with very similar interpretations.

In the 1983 American National Election Survey Pilot \citep{ANES}, each individual was prompted with various opinion-based statements and was asked to report their agreement with the statement using the categories: ``strongly agree", ``agree but not strongly", ``can't decide", ``disagree but not strongly", and ``strongly disagree". For example, one statement was ``Any person who is willing to work hard has a good chance of succeeding". We specifically study 19 of the statements which were selected by \cite{feldman1988structure} and reanalyzed by \citep{grossManriqueVallier}.  The 19 statements can be grouped into 3 overarching themes: Equality (abbreviated ``EQ" in the data set variable names), Economic Individualism (abbreviated ``IND") and Free Enterprise (abbreviated ``ENT") \citep{feldman1988structure}.  Following the original analysis of \cite{grossManriqueVallier}, we include the 279 complete responses and combine categories ``agree" with ``strongly agree" and ``disagree"  with ``strongly disagree", leaving the 3 possible responses ``agree" = 0, ``can't decide" = 1, and ``disagree" = 2 to avoid overparameterization. The data is included in \texttt{mixedMem} as \texttt{ANES}; the full text statement for each variable can be accessed through \texttt{help(ANES)}. 

A brief exploratory analysis, shows that of the $279 \times 19 = 5301$ total responses, 3295 responses are ``agree" 1907 are ``disagree" and only 99 are ``can't decide".


\begin{Schunk}
\begin{Sinput}
> library(mixedMem)
> data(ANES)
> # Dimensions of the data set: 279 individuals with 19 responses each
> dim(ANES)
\end{Sinput}
\begin{Soutput}
[1] 279  19
\end{Soutput}
\begin{Sinput}
> # The 19 variables and their categories
> # The specific statements for each variable can be found using help(ANES)
> # Variables titled EQ are about Equality
> # Variables titled IND are about Econonic Individualism
> # Variables titled ENT are about Free Enterprise
> colnames(ANES)
\end{Sinput}
\begin{Soutput}
 [1] "EQ1"  "EQ2"  "EQ3"  "EQ4"  "EQ5"  "EQ6"  "EQ7"  "IND1" "IND2" "IND3"
[11] "IND4" "IND5" "IND6" "ENT1" "ENT2" "ENT3" "ENT4" "ENT5" "ENT6"
\end{Soutput}
\begin{Sinput}
> # Distribution of responses
> table(unlist(ANES))
\end{Sinput}
\begin{Soutput}
   0    1    2 
3295   99 1907 
\end{Soutput}
\end{Schunk}

\subsection{Step 1: Initializing the \texttt{mixedMemModel} Object}
To fit a mixed membership model, we must first initialize a \texttt{mixedMemModel} object using the class constructor. The \texttt{mixedMemModel} object contains the dimensions of our mixed membership model, the observed data and initialization points for $\alpha$ and $\theta$. Creating a \texttt{mixedMemModel} object provides a vehicle for passing this information to the \texttt{mmVarFit} function in step 2. This is similar to how one might specify the formula for an \texttt{lm} object. Although initialization points for $\phi$ and $\delta$ can be passed to the constructor as well, these by default are initialized uniformly across the sub-populations, and unless there is very strong prior knowledge, we recommend that the default values be used. For this particular model, all the variables are multinomials; an example showing an initialization of mixed data types can be accessed through \texttt{help(mixedMemModel)}.

As mentioned previously, because the lower bound function is not convex, different initializations may result in convergence to different local modes. After initializing at various point, we found that the initialization of $\alpha = \left(.2, .2, .2\right)$ and $\theta \sim \text{Dirichlet}(.8)$ using seed 123 resulted in the highest lower bound at convergence.  

\begin{Schunk}
\begin{Sinput}
> # Sample Size
> Total <- 279
> # Number of variables
> J <- 19 
> # we only have one replicate for each of the variables
> Rj <- rep(1, J)
> # Nijr indicates the number of ranking levels for each variable.
> # Since all our data is multinomial it should be an array of all 1s
> Nijr <- array(1, dim = c(Total, J, max(Rj)))
> # Number of sub-populations
> K <- 3
> # There are 3 choices for each of the variables ranging from 0 to 2.
> Vj <- rep(3, J)
> # we initialize alpha to .2
> alpha <- rep(.2, K)
> # All variables are multinomial
> dist <- rep("multinomial", J)
> # obs are the observed responses. it is a 4-d array indexed by i,j,r,n
> # note that obs ranges from 0 to 2 for each response
> obs <- array(0, dim = c(Total, J, max(Rj), max(Nijr)))
> obs[, , 1, 1] <- as.matrix(ANES)
> # Initialize theta randomly with Dirichlet distributions
> set.seed(123)
> theta <- array(0, dim = c(J, K, max(Vj)))
> for (j in 1:J) {
+     theta[j, , ] <- gtools::rdirichlet(K, rep(.8, Vj[j]))
+ }
> # Create the mixedMemModel
> # This object encodes the initialization points for the variational EM algorithim
> # and also encodes the observed parameters and responses
> initial <- mixedMemModel(Total = Total, J = J, Rj = Rj,
+                          Nijr = Nijr, K = K, Vj = Vj, alpha = alpha,
+                          theta = theta, dist = dist, obs = obs)